% Notes and exercises on Finite Dimensional Vector Spaces
\documentclass[letterpaper,12pt]{article}
\usepackage{amsmath,amssymb,amsthm,enumitem,fourier}

\renewcommand{\ker}{\mathcal{N}}
\newcommand{\range}{\mathcal{R}}

\newcommand{\dsum}{\oplus}

\theoremstyle{definition}
\newtheorem*{exer}{Exercise}

\theoremstyle{remark}
\newtheorem*{rmk}{Remark}

\newtheoremstyle{direction}{0.5em}{0.5em}{}{}{}{}{0.5em}{}
\theoremstyle{direction}
\newtheorem*{fwd}{\(\implies\)}
\newtheorem*{bwd}{\(\impliedby\)}

% Meta
\title{\textit{Finite Dimensional Vector Spaces}\\Notes and Exercises}
\author{John Peloquin}
\date{}

\begin{document}
\maketitle

\section*{Chapter~II}
\subsection*{\S\ 49}
\begin{exer}[4]
Let \(V\)~be a vector space and \(E\)~and~\(F\) be projections on~\(V\).
\begin{enumerate}[itemsep=0pt]
\item[(a)] \(\range(E)=\range(F)\) if and only if \(EF=F\) and \(FE=E\).
\item[(b)] \(\ker(E)=\ker(F)\) if and only if \(EF=E\) and \(FE=F\).
\end{enumerate}
\end{exer}
\begin{proof}
Recall for a projection~\(P\) on~\(V\), \(V=\range(P)\dsum\ker(P)\) and (\S\ 41, Theorem 2)
\[\range(P)=\{\,x\in V\mid Px=x\,\}\qquad\ker(P)=\{\,x\in V\mid Px=0\,\}\]
\begin{enumerate}[itemsep=0pt]
\item[(a)]
\begin{fwd}
If \(x\in V\), then \(Ex\in\range(E)\subseteq\range(F)\), so \(FEx=F(Ex)=Ex\). Therefore \(FE=E\). Similarly \(EF=F\).
\end{fwd}
\begin{bwd}
If \(x\in\range(E)\), then \(x=Eu\) for some \(u\in V\), so
\[Fx=F(Eu)=FEu=Eu=x\]
and hence \(x\in\range(F)\). Therefore \(\range(E)\subseteq\range(F)\). Similarly \(\range(F)\subseteq\range(E)\) and hence \(\range(E)=\range(F)\).
\end{bwd}
\item[(b)]
\begin{fwd}
Since \(V=\range(E)\dsum\ker(E)\), if \(x\in V\) there exist \(u\in\range(E)\) and \(v\in\ker(E)\) with \(x=u+v\). Now
\begin{align*}
FEx&=FE(u+v)&&\\
	&=FEu+FEv&&\\
	&=Fu+F0&&\text{since \(u\in\range(E)\) and \(v\in\ker(E)\)}\\
	&=Fu+Fv&&\text{since \(\ker(E)\subseteq\ker(F)\)}\\
	&=F(u+v)\\
	&=Fx
\end{align*}
Therefore \(FE=F\). Similarly \(EF=E\).
\end{fwd}
\begin{bwd}
If \(x\in\ker(E)\), then
\[Fx=FEx=F(Ex)=F0=0\]
so \(x\in\ker(F)\). Therefore \(\ker(E)\subseteq\ker(F)\). Similarly \(\ker(F)\subseteq\ker(E)\) and hence \(\ker(E)=\ker(F)\).\qedhere
\end{bwd}
\end{enumerate}
\end{proof}
\begin{rmk}
By (a)~and~(b), \(E=F\) if and only if \(\range(E)=\range(F)\) and \(\ker(E)=\ker(F)\). In other words, projections are characterized by their ranges and null spaces.
\end{rmk}

\begin{exer}[5]
If \(E_1,\ldots,E_k\) are projections on~\(V\) with the same range and \(\alpha_1,\ldots,\alpha_k\) are scalars such that \(\sum_i\alpha_i=1\), then \(E=\sum_i\alpha_i E_i\) is a projection.
\end{exer}
\begin{proof}
By Exercise~4(a), we have
\begin{align*}
E^2&=\bigl(\,\sum_i\alpha_iE_i\bigr)^2&&\\
	&=\sum_i\sum_j\alpha_i\alpha_j E_iE_j&&\\
	&=\sum_i\sum_j\alpha_i\alpha_j E_j&&\text{since \(\range(E_i)=\range(E_j)\)}\\
	&=\bigl(\,\sum_i\alpha_i\bigr)\bigl(\,\sum_j\alpha_jE_j\bigr)&&\\
	&=1\cdot E&&\text{since \(\textstyle\sum_i\alpha_i=1\)}\\
	&=E
\end{align*}
Therefore \(E\)~is idempotent, and hence a projection (\S\ 41, Theorem~1).
\end{proof}

% References
\begin{thebibliography}{0}
\bibitem{halmos87} Halmos, P. \textit{Finite Dimensional Vector Spaces.} Springer, 1987.
\end{thebibliography}
\end{document}